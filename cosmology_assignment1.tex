\documentclass{article}
\usepackage[utf8]{inputenc}
\usepackage{cite}

\begin{document}

%%%%%%%%%%%%%%%%%%%%%%%%%%%%%%%%%%%%%%%%%%%%%%%%%%%%%%%%%%%%%%%%%%%%%%%%%%%
\section{What is the current expansion rate of the universe expressed via the Hubble constant, $H_0$?}

The LIGO and VIRGO Virgo gravitational wave detectors were recently able to make the first measurement of the Hubble constant in a manner completely independent of the cosmic distance ladder, as reported by Abbot et al. ~\cite{2017Natur.551...85A}. With the detection of the GW170817 event by LIGO and VIRGO and the detection of a corresponding gamma ray burst by traditional electromagnetic-spectrum telescopes, the era of multi-messenger astronomy was born and it became possible to use the gravitational wave in question as a standard siren, the gravitational equivalent of a standard candle. Using information from the gravitational wave, it is possible to determine the distance to the wave's source. Using electromagnetic information, recessional velocity can be calculated from the redshift of the source's spectra. Combining these two pieces of information allows researchers to make an estimate of the Hubble constant's value; in doing so, Abott et al. ~\cite{2017Natur.551...85A} obtained a value of $H_0 = 70.0^{+12.0}_{-8.0}$ km s$^{-1}$ Mpc$^{-1}$.

Word count: 168
 
%%%%%%%%%%%%%%%%%%%%%%%%%%%%%%%%%%%%%%%%%%%%%%%%%%%%%%%%%%%%%%%%%%%%%%%%%%%
\section{What is the current total normalised matter density of the universe, $\Omega_M$?}
Perlmetter et al. (1999) ~\cite{1999ApJ...517..565P} use type Ia supernovae (Sne Ia) as standard candles to make statistical claims about the joint probability density functions for cosmological parameters. In doing so, they obtain a value for normalized matter density of $\Omega_M = 0.28^{+0.09}_{-0.08}$, assuming a flat cosmology. The group used 42 SNe Ia to obtain these results and when comparing their magnitudes and redshifts, standardized the supernovae's magnitude peaks with a known SNe Ia light-curve width-luminosity relation in order to give the most statistically reasonable results. It has been proven possible to separate the relative contribution of mass density from that of the cosmological constant using this method, making it possible to determine $\Omega_M$ through a survey of high-redshift Sne Ia. The Supernova Cosmology Project, from which the sample used in this analysis was drawn, had already observed over 75 Sne Ia at redshifts $z = 0.18?0.86$ as of 1998; by now, many more have been observed, creating a large database from which to make even more precise measurements.

Word count: 182

%%%%%%%%%%%%%%%%%%%%%%%%%%%%%%%%%%%%%%%%%%%%%%%%%%%%%%%%%%%%%%%%%%%%%%%%%%%
\section{What are the geometric properties of the universe: open, flat or closed?}

The standard cosmology which most researchers assume is called $\Lambda$CDM and describes a spatially flat universe. The Planck mission, a space-based observatory which was tasked with observing the cosmic microwave background (CMB) made measurements of the temperature power spectrum which were highly consistent with the standard $\Lambda$CDM cosmology  ~\cite{2016A&A...594A..13P}. The authors conclude that our universe is tightly constrained in being spatially flat or very-nearly spatially flat; they report that the universe is flat to 1$\sigma$ uncertainty level of 0.25 percent. Making curvature calculations, a value of $\Omega_k = 0.000 \pm{0.005}$ is reported, indicating that the universe is, indeed, highly likely to be spatially flat.

Word count: 116
%%%%%%%%%%%%%%%%%%%%%%%%%%%%%%%%%%%%%%%%%%%%%%%%%%%%%%%%%%%%%%%%%%%%%%%%%%%

\bibliographystyle{plain}
\bibliography{biblio.bib}

\end{document}